%---------- Inleiding ---------------------------------------------------------

\section{Introductie}%
\label{sec:introductie}

Bedrijven merken het laatste jaar op dat hun maandelijkse energiefactuur begint te stijgen, een van de reden hiervoor is dat bedrijven veel vaker elektrische wagens aankopen, omdat deze aankoop fiscaal voordeliger is. Elektrische wagens zijn, momenteel, namenlijk 100\% fiscaal aftrekbaar volgens \textcite{Blomme2023}. De tweede reden is een gevolg van het aankopen van elektrische wagens, zo kunnen bedrijven ook laadpalen voor deze wagens aankopen. Dit gebeurt zodat de werknemers hun wagens op de plaats waar het bedrijf gevestigd is, kunnen laden. Vooral de tweede reden heeft invloed op de bedrijven, omdat hoe meer laadpalen er gebruikt worden, hoe meer elektriciteit er verbruikt wordt. Bovenop de vorige twee redenen, dienen bedrijven vanaf werkjaar 2023 een toegangsvermogen te bepalen. Dit toegangsvermogen uitgedrukt in kilowatt, is een eigen inschatting van je maandelijkse piekvermogen \autocite{Fluvius2022}. Het maandelijkse piekvermogen bevat het kwartier in de maand met het hoogste verbruik, anders verwoord is dit het hoogste gemiddelde vermogen, gemeten in een kwartier \autocite{Fluvius2022}. Volgens \textcite{Fluvius2022} wordt er een boete aangerekend waneer het werkelijk maandelijkse piekvermogen hoger ligt dan de eigen opgegeven toegangsvermogen. Dit bedrag is gebaseerd op het verschil tussen de te hoge piekvermogen en het doorgegeven toegangsvermogen. Die waarde in kilowatt wordt dan 12 maanden lang aangerekend volgens een tarief dat 50\% hoger ligt dan het toegangsvermogen.

Dit geldt ook voor Carwash Clean Car, een kleine zelfstandige, gelegen te Dendermonde. De eigenaar van deze carwash zoekt hierdoor een oplossing om zijn maandelijkse energiefactuur te verlagen. De eigenaar heeft al een aantal maatregelen genomen om de energiefactuur te verlagen, zo zijn er al zonnepanelen geplaatst, de auto's worden enkel ingestoken als de zon schijnt of na 22 uur 's avonds, omdat er dan gebruik wordt gemaakt van het nachttarief... Maar de wagens enkel laden als de zon schijnt helpt niet echt om de energiefactuur te verlagen, omdat soms vergeten wordt de lader los te koppelen van de wagen. Daarom wordt er in deze bachelorproef onderzoek gedaan om de vraag te beantwoorden: "Hoe kan het maandelijks elektrisiteitsverbruik van het bedrijf Carwash Clean Car gemonitord worden om zo het verbruik van de laadpalen voor auto's te optimaliseren door middel van het slim aansturen van deze laadpalen met behulp van een custom geschreven applicatie?".

Als het onderzoek succesvol is, kan de eigenaar van Carwash Clean Car zijn laadpalen voor auto's laten aansturen via een custom geschreven applicatie om zo het energieverbruik te optimaliseren. Dit onderzoek zal hier een antwoord op bieden aan de hand van een literatuurstudie, waar er verschillende technologieën worden bekeken, welke connecties moeten gemaakt worden om de juiste dataregisters aan te spreken van de laadpalen en monitoring van het energieverbruik. Na de literatuurstudie wordt er een proefopstelling opgesteld om de gevonden informatie toe te passen, om zo een progressive web app te maken voor het bedrijf.

%---------- Stand van zaken ---------------------------------------------------

\section{State-of-the-art}%
\label{sec:state-of-the-art}

Hier beschrijf je de \emph{state-of-the-art} rondom je gekozen onderzoeksdomein, d.w.z.\ een inleidende, doorlopende tekst over het onderzoeksdomein van je bachelorproef. Je steunt daarbij heel sterk op de professionele \emph{vakliteratuur}, en niet zozeer op populariserende teksten voor een breed publiek. Wat is de huidige stand van zaken in dit domein, en wat zijn nog eventuele open vragen (die misschien de aanleiding waren tot je onderzoeksvraag!)?

Je mag de titel van deze sectie ook aanpassen (literatuurstudie, stand van zaken, enz.). Zijn er al gelijkaardige onderzoeken gevoerd? Wat concluderen ze? Wat is het verschil met jouw onderzoek?


Draag zorg voor correcte literatuurverwijzingen! Een bronvermelding hoort thuis \emph{binnen} de zin waar je je op die bron baseert, dus niet er buiten! Maak meteen een verwijzing als je gebruik maakt van een bron. Doe dit dus \emph{niet} aan het einde van een lange paragraaf. Baseer nooit teveel aansluitende tekst op eenzelfde bron.

Als je informatie over bronnen verzamelt in JabRef, zorg er dan voor dat alle nodige info aanwezig is om de bron terug te vinden (zoals uitvoerig besproken in de lessen Research Methods).

% Voor literatuurverwijzingen zijn er twee belangrijke commando's:
% \autocite{KEY} => (Auteur, jaartal) Gebruik dit als de naam van de auteur
%   geen onderdeel is van de zin.
% \textcite{KEY} => Auteur (jaartal)  Gebruik dit als de auteursnaam wel een
%   functie heeft in de zin (bv. ``Uit onderzoek door Doll & Hill (1954) bleek
%   ...'')

Je mag deze sectie nog verder onderverdelen in subsecties als dit de structuur van de tekst kan verduidelijken.

%---------- Methodologie ------------------------------------------------------
\section{Methodologie}%
\label{sec:methodologie}

Hier beschrijf je hoe je van plan bent het onderzoek te voeren. Welke onderzoekstechniek ga je toepassen om elk van je onderzoeksvragen te beantwoorden? Gebruik je hiervoor literatuurstudie, interviews met belanghebbenden (bv.~voor requirements-analyse), experimenten, simulaties, vergelijkende studie, risico-analyse, PoC, \ldots?

Valt je onderwerp onder één van de typische soorten bachelorproeven die besproken zijn in de lessen Research Methods (bv.\ vergelijkende studie of risico-analyse)? Zorg er dan ook voor dat we duidelijk de verschillende stappen terug vinden die we verwachten in dit soort onderzoek!

Vermijd onderzoekstechnieken die geen objectieve, meetbare resultaten kunnen opleveren. Enquêtes, bijvoorbeeld, zijn voor een bachelorproef informatica meestal \textbf{niet geschikt}. De antwoorden zijn eerder meningen dan feiten en in de praktijk blijkt het ook bijzonder moeilijk om voldoende respondenten te vinden. Studenten die een enquête willen voeren, hebben meestal ook geen goede definitie van de populatie, waardoor ook niet kan aangetoond worden dat eventuele resultaten representatief zijn.

Uit dit onderdeel moet duidelijk naar voor komen dat je bachelorproef ook technisch voldoen\-de diepgang zal bevatten. Het zou niet kloppen als een bachelorproef informatica ook door bv.\ een student marketing zou kunnen uitgevoerd worden.

Je beschrijft ook al welke tools (hardware, software, diensten, \ldots) je denkt hiervoor te gebruiken of te ontwikkelen.

Probeer ook een tijdschatting te maken. Hoe lang zal je met elke fase van je onderzoek bezig zijn en wat zijn de concrete \emph{deliverables} in elke fase?

%---------- Verwachte resultaten ----------------------------------------------
\section{Verwacht resultaat, conclusie}%
\label{sec:verwachte_resultaten}

Hier beschrijf je welke resultaten je verwacht. Als je metingen en simulaties uitvoert, kan je hier al mock-ups maken van de grafieken samen met de verwachte conclusies. Benoem zeker al je assen en de onderdelen van de grafiek die je gaat gebruiken. Dit zorgt ervoor dat je concreet weet welk soort data je moet verzamelen en hoe je die moet meten.

Wat heeft de doelgroep van je onderzoek aan het resultaat? Op welke manier zorgt jouw bachelorproef voor een meerwaarde?

Hier beschrijf je wat je verwacht uit je onderzoek, met de motivatie waarom. Het is \textbf{niet} erg indien uit je onderzoek andere resultaten en conclusies vloeien dan dat je hier beschrijft: het is dan juist interessant om te onderzoeken waarom jouw hypothesen niet overeenkomen met de resultaten.

