\chapter{\IfLanguageName{dutch}{Proefopstelling}{Proof of concept}}
\label{ch:proefopstelling}

\section{Inleiding}
\label{sec:proefopstelling-inleiding}

In dit deel van de bachelorproef wordt er besproken, hoe de proefopstelling opgezet is, welke problemen zich voordoen tijdens het opzetten van de proefopstelling, hoe deze problemen opgelost zijn, welke problemen er zich voordoen tijdens het ontwikkelen van de applicatie en hoe deze opgelost zijn geweest. Het einddoel van dit hoofdstuk is een werkende applicatie te hebben.

\section{Installatieproces van alle nodige software}
\label{sec:proefopstelling-opzetten}

Het eerste wat gebeurd is, is het installeren van het besturingssysteem op de Raspberry Pi. Het besturingssysteem op de SD-kaart zetten ging vlot aangezien het team van Raspberry Pi, hier zelf een tool voor heeft gecreëerd. In deze tool kan er gekozen worden op welk Raspberry Pi model het besturingssysteem geïnstalleerd moet worden, zo worden alle besturingssystemen weergeven die compatibel zijn met het gekozen model. Na het kiezen van het besturingssysteem zijn er al instellingen ingesteld geweest via de tool, zo is er al een standaard WIFI-connectie opgezet, een gebruiker aangemaakt en is het SSH-protocol aangezet geweest.\\

De volgende stap was het installeren van node-red, deze installatie is vlot verlopen, aangezien er op de website van node-red een commando staat voor de commandline die alle nodige software installeert voor een basis node-red installatie \footnote{Het commando kan teruggevonden worden op deze website: \url{https://nodered.org/docs/getting-started/raspberrypi}. Deze link is bezocht op 14 mei 2024.}.\\

De laatste stap van het installeren van alle nodige software was het aanmaken van de React JS applicatie, deze applicatie maakt gebruik van het Next JS framework met tailwind CSS. Dit is de basis van de webapplicatie, tijdens het ontwikkelen worden er nog extra packages geïnstalleerd om het ontwikkelen van de applicatie makkelijker te maken.

\section{Ontwikkelen van de backend}
\label{sec:proefopstelling-backend}

De backend bestaat uit verschillende flows met elk hun eigen functionaliteit. Door het project op te delen in verschillende flows is het makkelijk te onderscheiden waar welke functionaliteit staat.

\subsection{Flow 1: Authenticatie}
\label{subsec:proefopstelling-authenticatie}

De eerste flow wordt gebruikt voor het inloggen en aanmaken van een gebruiker. Het inloggen van een gebruiker gebeurt aan de hand van het ophalen van de gebruiker informatie uit de databank met het mailadres dat wordt doorgegeven. Als er een mailadres van een gebruiker overeenkomt met het doorgegeven mailadres, dan wordt de gebruikersinformatie teruggegeven. Zo kan de webapplicatie kijken of het gehashte wachtwoord overeenkomt met het opgegeven wachtwoord.

\subsection{Flow 2: Electricity}
\label{subsec:proefopstelling-electricity}

Deze flow wordt gebruikt voor het ophalen van het huidige energieverbruik, de huidige energieopbrengst en de huidige aankoop van energie. Zo worden al deze waarden teruggegeven in een JSON object. In deze flow worden ook berekeningen uitgevoerd zodat de waarden in het juiste formaat worden weergegeven.

\subsection{Flow 3: Charging Station}
\label{subsec:proefopstelling-charging-station}

In deze flow worden alle requests van de laadpaal afgehandeld. Zo wordt voor elke GET-request de waarde opgehaald via het modbus protocol en opgeslagen in een globale variabele. De waarde van de globale variabele wordt dan opgeroepen en in een JSON object gestoken, zodat dit JSON object teruggestuurd kan worden.\\

De POST-requests krijgen een body mee, vanuit de webapplicatie, aan de hand van deze body wordt de juiste waarde doorgegeven aan de laadpaal zijn register. Zodat de laadpaal aangestuurd kan worden.

\subsection{Flow 4: Blower}
\label{subsec:proefopstelling-blower}

Deze flow dient voor het aansturen van het beluchtingssysteem van het water. Zo zijn er 2 requests die afgehandeld moeten worden, namelijk 1 GET-request en 1 POST-request. De GET-request wordt gebruikt om te weten of het beluchtingssysteem ingeschakeld is.\\

De POST-request krijgt een body mee, vanuit de webapplicatie, de waarde in deze body is een booleaanse waarde. Aan de hand van deze booleaanse waarde kan het beluchtingsysteem dan ingeschakeld worden.

\subsection{Flow 5: Algoritmen}
\label{subsec:proefopstelling-algoritmen}

De laatste flow bevat de algoritmen om de bedrijfsprocessen automatisch aan te sturen. In deze flow zijn er 2 algoritmen aanwezig, het eerste algoritme zal de prioriteiten van de bedrijfsprocessen bepalen en in de databank stockeren. Nadien kan het tweede algoritme bepalen welke bedrijfsprocessen ingeschakeld kunnen worden, en deze dan inschakelen. 

\section{Ontwikkelen van de webapplicatie}
\label{sec:proefopstelling-webapplicatie}

De webapplicatie is ontwikkeld in React JS. De styling van de applicatie is gebeurd met Tailwind CSS. In de broncode van de applicatie worden verschillende thema packages gebruikt, de meest gebruikte packages zijn ShadCn UI en ChartJS. ShadCn UI is een libary met verschillende componenten, deze componenten kunnen verder gepersonaliseerd worden met Tailwind CSS. ChartJS wordt gebruikt om de grafiek van het elektriciteitsverbruik te genereren.\\