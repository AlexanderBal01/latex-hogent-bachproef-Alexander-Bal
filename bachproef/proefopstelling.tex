\chapter{\IfLanguageName{dutch}{Proefopstelling}{Proof of concept}}
\label{ch:proefopstelling}

\section{Inleiding}
\label{sec:proefopstelling-inleiding}

In dit deel van de bachelorproef wordt er besproken, hoe de proefopstelling opgezet is, welke problemen zich voordoen tijdens het opzetten van de proefopstelling, hoe deze problemen opgelost zijn, welke problemen er zich voordoen tijdens het ontwikkelen van de applicatie en hoe deze opgelost zijn geweest. Het einddoel van dit hoofdstuk is een werkende applicatie te hebben.

\section{Installatieproces van alle nodige software}
\label{sec:proefopstelling-opzetten}

Het eerste wat gebeurd is, is het installeren van het besturingssysteem op de Raspberry Pi. Het besturingssysteem op de SD-kaart zetten ging vlot aangezien het team van Raspberry Pi, hier zelf een tool voor heeft gecreëerd. In deze tool kan er gekozen worden op welk Raspberry Pi model het besturingssysteem geïnstalleerd moet worden, zo worden alle besturingssystemen weergeven die compatibel zijn met het gekozen model. Na het kiezen van het besturingssysteem zijn er al instellingen ingesteld geweest via de tool, zo is er al een standaard WIFI-connectie opgezet, een gebruiker aangemaakt en is het SSH-protocol aangezet geweest.\\

De volgende stap was het installeren van node-red, deze installatie is vlot verlopen, aangezien er op de website van node-red een commando staat voor de commandline die alle nodige software installeert voor een basis node-red installatie \footnote{Het commando kan teruggevonden worden op deze website: \url{https://nodered.org/docs/getting-started/raspberrypi}. Deze link is bezocht op 14 mei 2024.}.\\

De laatste stap van het installeren van alle nodige software was het aanmaken van de React JS applicatie, deze applicatie maakt gebruik van het Next JS framework met tailwind CSS. Dit is de basis van de webapplicatie, tijdens het ontwikkelen worden er nog extra packages geïnstalleerd om het ontwikkelen van de applicatie makkelijker te maken.

\section{Ontwikkelen van de backend}
\label{sec:proefopstelling-backend}