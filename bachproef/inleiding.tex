%%=============================================================================
%% Inleiding
%%=============================================================================

\chapter{\IfLanguageName{dutch}{Inleiding}{Introduction}}%
\label{ch:inleiding}

De regel van het piekvermogen geldt voor alle bedrijven in België, dus ook voor Carwash Clean Car, een kleine zelfstandige, gelegen te Dendermonde. De eigenaar van deze carwash zoekt hierdoor naar een oplossing om de maandelijkse energiefactuur te verlagen. Zo heeft de eigenaar al een aantal maatregelen getroffen om de energiefactuur te verlagen. Er werden al zonnepanelen geplaatst en de eigen elektrische voertuigen werden enkel opgeladen bij voldoende zonlicht of na 22 uur, omdat er dan wordt gebruikgemaakt van het nachttarief.

\pagebreak

\section{\IfLanguageName{dutch}{Probleemstelling}{Problem Statement}}%
\label{sec:probleemstelling}

Bedrijven kopen steeds vaker elektrische wagens aan, omdat deze aankoop fiscaal voordeliger is tegenover de aankoop van wagens die rijden op fossiele brandstoffen. Dit komt doordat de Belgische overheid heeft beslist dat tegen 2026 bedrijven enkel nog met elektrische bedrijfsvoertuigen mogen rondrijden. Ten gevolge van de aankoop van elektrische voertuigen plaatsen bedrijven laadpalen, zodat de medewerkers hun wagen in het bedrijf kunnen opladen. Dit zorgt ervoor dat bedrijven hun maandelijkse energiefactuur zien stijgen, want hoe meer laadpalen er gebruikt worden hoe meer elektriciteit er verbruikt wordt. Bedrijven dienen ook vanaf werkjaar 2023 een toegangsvermogen te bepalen en door te geven aan de netbeheerder. Dit toegangsvermogen uitgedrukt in kilowatt, is een eigen inschatting van het maandelijkse piekvermogen. Het maandelijkse piekvermogen wordt bepaald door het kwartier in de maand met het hoogste verbruik, anders verwoordt dit het hoogste gemiddelde vermogen, gemeten in een kwartier, in de maand. Er wordt een boete aangerekend wanneer het werkelijk maandelijks piekvermogen hoger ligt dan het eigen opgegeven toegangsvermogen. Deze bachelorproef onderzoekt hoe bepaalde bedrijven hun maandelijkse energiefactuur kunnen verlagen door het slim aansturen van de laadpalen en elektriciteit-intensieve bedrijfsprocessen.

\section{\IfLanguageName{dutch}{Onderzoeksvraag}{Research question}}%
\label{sec:onderzoeksvraag}

\subsection{\IfLanguageName{dutch}{Hoofdonderzoeksvraag}{Main research question}}
\label{subsec:hoofdonderzoeksvraag}

In sectie \ref{sec:probleemstelling} werd de hoofdonderzoeksvraag al subtiel vermeld, daar werd vermeld dat deze bachelorproef zich zal focussen op het onderzoek van, hoe het bedrijf Carwash Clean Car zijn maandelijkse energiefactuur kan verlagen door het slim aansturen van de laadpalen en elektriciteit-intensieve bedrijfsprocessen. Zo kunnen deze omstandigheden samengevoegd worden tot één onderzoeksvraag:

\begin{itemize}
  \item Hoe kan het bedrijf Carwash Clean Car zijn maandelijkse energiefactuur verlagen door het slim aansturen van laadpalen en elektriciteit-intensieve bedrijfsprocessen aan de hand van een custom geschreven applicatie?
\end{itemize}

\pagebreak

\subsection{\IfLanguageName{dutch}{Deelonderzoeksvragen}{Subquestions}}
\label{subsec:deelonderzoeksvragen}

Om de hoofdonderzoeksvraag te beantwoorden, zal deze opgesplitst worden in deelonderzoeksvragen. Deze deelonderzoeksvragen zullen de basis vormen voor de literatuurstudie en het onderzoek. De deelonderzoeksvragen zijn:

\begin{itemize}
  \item \textbf{Probleemdomein:}
        \begin{itemize}
          \item Wat zijn de belangrijkste energieverbruikers binnen Carwash Clean Car?
          \item Welke factoren beïnvloeden de energiekosten van het bedrijf?
          \item Op welke manier kan het gebruik van laadpalen voor elektrische voertuigen worden geoptimaliseerd om piekbelastingen te verminderen en kosten te besparen?
          \item Welke technologische oplossingen zijn beschikbaar voor het slim aansturen van laadpalen en elektriciteit-intensieve bedrijfsprocessen?
          \item Wat zijn de potentiële voordelen en uitdagingen van het implementeren van een slim energiebeheersysteem in een bedrijf als Carwash Clean Car?
          \item Zijn er best practices of case studies van vergelijkbare bedrijven die succesvol energiebesparende maatregelen hebben geïmplementeerd die relevant zijn voor Carwash Clean Car?
        \end{itemize}
  \item  \textbf{oplossings domein:}
        \begin{itemize}
          \item Hoe kan bidirectioneel laden handig zijn binnen deze casus?
          \item Welke informatie is er nodig om aan de hand van een algoritme te bepalen welk bedrijfsproces voorrang heeft?
          \item Welke technologieën zijn er beschikbaar om laadpalen en elektriciteit-intensieve bedrijfsprocessen aan te sturen?
          \item Welke technologieën zijn er beschikbaar om het energieverbruik van laadpalen en elektriciteit-intensieve bedrijfsprocessen te monitoren?
        \end{itemize}
\end{itemize}

Doorheen de bachelorproef zullen deze deelonderzoeksvragen beantwoord worden, zodat de hoofdonderzoeksvraag beantwoord kan worden in hoofdstuk \ref{ch:conclusie}.

\section{\IfLanguageName{dutch}{Onderzoeksdoelstelling}{Research objective}}%
\label{sec:onderzoeksdoelstelling}

Het beoogde resultaat van deze bachelorproef zal een custom webapplicatie zijn die het bedrijf Carwash Clean Car zal helpen om bepaalde bedrijfsprocessen aan te sturen en het energieverbruik te monitoren. Natuurlijk geeft deze applicatie ook inzicht in het energieverbruik van de laadpalen en de elektriciteit-intensieve bedrijfsprocessen, zo kan de eigenaar van Carwash Clean Car deze processen bijsturen om de maandelijkse energiefactuur te verlagen.

\section{\IfLanguageName{dutch}{Opzet van deze bachelorproef}{Structure of this bachelor thesis}}%
\label{sec:opzet-bachelorproef}

% Het is gebruikelijk aan het einde van de inleiding een overzicht te
% geven van de opbouw van de rest van de tekst. Deze sectie bevat al een aanzet
% die je kan aanvullen/aanpassen in functie van je eigen tekst.

De rest van deze bachelorproef is als volgt opgebouwd:\\

In Hoofdstuk~\ref{ch:stand-van-zaken} wordt een overzicht gegeven van de stand van zaken binnen het onderzoeksdomein, op basis van een literatuurstudie.\\

In Hoofdstuk~\ref{ch:methodologie} wordt de methodologie toegelicht en worden de gebruikte onderzoekstechnieken besproken om een antwoord te kunnen formuleren op de onderzoeksvragen.\\

In Hoofdstuk~\ref{ch:proefopstelling} wordt de proefopstelling besproken en wordt de custom applicatie toegelicht.\\

In Hoofdstuk~\ref{ch:conclusie}, tenslotte, wordt de conclusie gegeven en een antwoord geformuleerd op de onderzoeksvragen. Daarbij wordt ook een aanzet gegeven voor toekomstig onderzoek binnen dit domein.