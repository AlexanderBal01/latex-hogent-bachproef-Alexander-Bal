%%=============================================================================
%% Inleiding
%%=============================================================================

\chapter{\IfLanguageName{dutch}{Inleiding}{Introduction}}%
\label{ch:inleiding}

\section{\IfLanguageName{dutch}{Context, achtergrond}{Context, background}}%
\label{sec:context}

Bedrijven kopen steeds vaker elektrische wagens aan, omdat deze aankoop fiscaal voordeliger is tegenover de aankoop van wagens die rijden op fosiele brandstoffen. Ten gevolge van deze aankoop, plaatsen bedrijven laadpalen, zodat de medewerkers hun wagen in het bedrijf kunnen opladen. Dit zorgt ervoor dat bedrijven hun maandelijkse energiefactuur zien stijgen, want hoe meer laadpalen er gebruikt worden hoe meer elektriciteit er verbruikt wordt. Bovenop de vorige reden, dienen bedrijven vanaf werkjaar 2023 een toegangsvermogen te bepalen en door te geven aan de netbeheerder. Dit toegangsvermogen uitgedrukt in kilowatt, is een eigen inschatting van het maandelijkse piekvermogen. Het maandelijkse piekvermogen wordt bepaald door het kwartier in de maand met het hoogste verbruik, anders verwoord is dit het hoogste gemiddelde vermogen, gemeten in een kwartier, in de maand. Er wordt een boete aangerekend wanneer het werkelijk maandelijks piekvermogen hoger ligt dan het eigen opgegeven toegangsvermogen. Zo wordt er onderzocht hoe bepaalde bedrijven hun maandelijkse energiefactuur kunnen verlagen door het slim aansturen van de laadpalen en elektriciteits intensieve bedrijfsprocessen.

\section{\IfLanguageName{dutch}{afbakenen van het onderwerp}{Delimitation of the subject}}%
\label{sec:afbakening}

De regel van het piekvermogen geldt ook voor Carwash Clean Car, een kleine zelfstandige, gelegen te dendermonde. De eigenaar van deze carwash zoekt hierdoor naar een oplossing om de maandelijkse energiefactuur te verlagen. De eigenaar heeft al een aantal maatregelen getroffen om de energiefactuur te verlagen. Zo werden er al zonnepanelen geplaats en de eigen voertuigen worden enkel opgeladen bij voldoende zonlicht of na 22 uur, omdat er dan gebruik wordt gemaakt van het nachttarief. 

\begin{itemize}
  \item context, achtergrond
  \item afbakenen van het onderwerp
  \item verantwoording van het onderwerp, methodologie
  \item probleemstelling
  \item onderzoeksdoelstelling
  \item onderzoeksvraag
  \item \ldots
\end{itemize}

\section{\IfLanguageName{dutch}{Probleemstelling}{Problem Statement}}%
\label{sec:probleemstelling}

Uit je probleemstelling moet duidelijk zijn dat je onderzoek een meerwaarde heeft voor een concrete doelgroep. De doelgroep moet goed gedefinieerd en afgelijnd zijn. Doelgroepen als ``bedrijven,'' ``KMO's'', systeembeheerders, enz.~zijn nog te vaag. Als je een lijstje kan maken van de personen/organisaties die een meerwaarde zullen vinden in deze bachelorproef (dit is eigenlijk je steekproefkader), dan is dat een indicatie dat de doelgroep goed gedefinieerd is. Dit kan een enkel bedrijf zijn of zelfs één persoon (je co-promotor/opdrachtgever).

\section{\IfLanguageName{dutch}{Onderzoeksvraag}{Research question}}%
\label{sec:onderzoeksvraag}

Wees zo concreet mogelijk bij het formuleren van je onderzoeksvraag. Een onderzoeksvraag is trouwens iets waar nog niemand op dit moment een antwoord heeft (voor zover je kan nagaan). Het opzoeken van bestaande informatie (bv. ``welke tools bestaan er voor deze toepassing?'') is dus geen onderzoeksvraag. Je kan de onderzoeksvraag verder specifiëren in deelvragen. Bv.~als je onderzoek gaat over performantiemetingen, dan

\section{\IfLanguageName{dutch}{Onderzoeksdoelstelling}{Research objective}}%
\label{sec:onderzoeksdoelstelling}

Wat is het beoogde resultaat van je bachelorproef? Wat zijn de criteria voor succes? Beschrijf die zo concreet mogelijk. Gaat het bv.\ om een proof-of-concept, een prototype, een verslag met aanbevelingen, een vergelijkende studie, enz.

\section{\IfLanguageName{dutch}{Opzet van deze bachelorproef}{Structure of this bachelor thesis}}%
\label{sec:opzet-bachelorproef}

% Het is gebruikelijk aan het einde van de inleiding een overzicht te
% geven van de opbouw van de rest van de tekst. Deze sectie bevat al een aanzet
% die je kan aanvullen/aanpassen in functie van je eigen tekst.

De rest van deze bachelorproef is als volgt opgebouwd:

In Hoofdstuk~\ref{ch:stand-van-zaken} wordt een overzicht gegeven van de stand van zaken binnen het onderzoeksdomein, op basis van een literatuurstudie.

In Hoofdstuk~\ref{ch:methodologie} wordt de methodologie toegelicht en worden de gebruikte onderzoekstechnieken besproken om een antwoord te kunnen formuleren op de onderzoeksvragen.

% TODO: Vul hier aan voor je eigen hoofstukken, één of twee zinnen per hoofdstuk

In Hoofdstuk~\ref{ch:conclusie}, tenslotte, wordt de conclusie gegeven en een antwoord geformuleerd op de onderzoeksvragen. Daarbij wordt ook een aanzet gegeven voor toekomstig onderzoek binnen dit domein.