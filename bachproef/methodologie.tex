%%=============================================================================
%% Methodologie
%%=============================================================================

\chapter{\IfLanguageName{dutch}{Methodologie}{Methodology}}%
\label{ch:methodologie}

%% TODO: In dit hoofstuk geef je een korte toelichting over hoe je te werk bent
%% gegaan. Verdeel je onderzoek in grote fasen, en licht in elke fase toe wat
%% de doelstelling was, welke deliverables daar uit gekomen zijn, en welke
%% onderzoeksmethoden je daarbij toegepast hebt. Verantwoord waarom je
%% op deze manier te werk gegaan bent.
%% 
%% Voorbeelden van zulke fasen zijn: literatuurstudie, opstellen van een
%% requirements-analyse, opstellen long-list (bij vergelijkende studie),
%% selectie van geschikte tools (bij vergelijkende studie, "short-list"),
%% opzetten testopstelling/PoC, uitvoeren testen en verzamelen
%% van resultaten, analyse van resultaten, ...
%%
%% !!!!! LET OP !!!!!
%%
%% Het is uitdrukkelijk NIET de bedoeling dat je het grootste deel van de corpus
%% van je bachelorproef in dit hoofstuk verwerkt! Dit hoofdstuk is eerder een
%% kort overzicht van je plan van aanpak.
%%
%% Maak voor elke fase (behalve het literatuuronderzoek) een NIEUW HOOFDSTUK aan
%% en geef het een gepaste titel.

\section{Literatuurstudie}
\label{sec:methodologie-literatuurstudie}

Om het onderzoek te starten wordt er eerst een literatuurstudie uitgevoerd. De literatuurstudie is opgedeeld in 3 delen. Elk deel van de literatuurstudie focust zich op een klein onderdeel van de onderzoeksvraag. Eens een deel is afgewerkt, komt hier een lijst uit voort met mogelijke oplossingen die gebruikt kunnen worden in het volgende deel.

\subsection{Deel 1}
\label{subsec:methodologie-literatuurstudie-deel1}

Het eerste deel bestaat uit het bepalen van het energieverbruik van de belangrijkste bedrijfsprocessen en de niet-dringende bedrijfsprocessen, om dan concepten en methoden te bespreken die het gebruikt kunnen worden om het energieverbruik efficiënter te laten verlopen, dat zo de energiefactuur verlaagt.

\subsection{Deel 2}
\label{subsec:methodologie-literatuurstudie-deel2}

In het tweede deel wordt namelijk onderzocht hoe de gevonden concepten en methoden kunnen toegepast worden op de niet-dringende bedrijfsprocessen. Op het einde van dit deel is er bekend welke bedrijfsprocessen en concepten en methoden in de proefopstelling gebruikt kunnen worden.

\subsection{Deel 3}
\label{subsec:methodologie-literatuurstudie-deel3}

Het laatste deel van deze studie bestaat uit het onderzoeken van de samenwerking van de componenten en gevonden concepten en methoden die gebruikt worden in de proefopstelling. Zodat het ontwikkelen van de proefopstelling vlot te werk gaat.

\section{Analyse van de vereisten}
\label{sec:methodologie-analyse-vereisten}

In dit hoofdstuk worden de vereisten vastgelegd voor de proefopstelling, dit gebeurt aan de hand van een requirementsanalyse. Aan de hand van deze vereisten wordt dan afgewogen welke componenten zeker in de proefopstelling verwerkt moeten zitten.

\section{Proefopstelling}
\label{sec:methodologie-proefopstelling}

Na het hoofdstuk van de  analyse van de vereisten wordt er een proefopstelling ontwikkeld met de gekozen bedrijfsprocessen en de gekozen concepten en methoden. Hierbij wordt een webapplicatie ontwikkeld die minimum volgende functionaliteit moet bevatten:

\begin{itemize}
    \item Het inloggen en uitloggen van een gebruiker.
    \item Het aansturen van de bedrijfsprocessen.
    \item Het weergeven van het huidige energieverbruik.
\end{itemize}

De proefopstelling bevat ook backend applicatie die de volgende verzoeken van de webapplicatie afhandeld:

\begin{itemize}
    \item Het inloggen en uitloggen van een gebruiker.
    \item De aansturing van de bedrijfsprocessen, volgend uit de verzoeken van de webapplicatie.
    \item De data opvragen voor de weergave van het huidige energieverbruik
\end{itemize}

De combinatie van deze 2 applicaties zorgt dan voor het eindproduct van deze fase.

\section{Testen van de proefopstelling}
\label{sec:methodologie-testen-proefopstelling}

De volgende stap in de bachelorproef is het testen en verbeteren van de bachelorproef. Dit is een herhalend proces, zodat de proefopstelling zeker voldoet aan de verwachtingen van het bedrijf en de doelstellingen van deze bachelorproef. Het eindproduct van deze fase is een lijst van doorgevoerde verbeteringen. Deze verbeteringen worden dan besproken in dit hoofdstuk.

\section{Conclusie}
\label{sec:methodologie-conclusie}

De laatste fase van deze bachelorproef is het schrijven van een conclusie. Hier wordt de finale versie van de proefopstelling en de verbeteringen tegenover vorige versies besproken. Er worden ook oplossingen van methoden aangeboden die in de literatuurstudie besproken zijn, maar niet binnen de proefopstelling passen. Het eindproduct van deze fase is een conclusie over de finale versie proefopstelling en een advies van aanpassingen die geïmplementeerd kunnen worden om het verbruik van het bedrijf nog verder te optimaliseren.