\chapter{\IfLanguageName{dutch}{Stand van zaken}{State of the art}}%
\label{ch:stand-van-zaken}

% Tip: Begin elk hoofdstuk met een paragraaf inleiding die beschrijft hoe
% dit hoofdstuk past binnen het geheel van de bachelorproef. Geef in het
% bijzonder aan wat de link is met het vorige en volgende hoofdstuk.

% Pas na deze inleidende paragraaf komt de eerste sectiehoofding.

\section{Inleiding}
\label{sec:stand-van-zaken-inleiding}

In dit deel van de bachelor proef is er een uitgebreid onderzoek uitgevoerd. Zo worden er verschillende onderdelen besproken die relevant zijn voor deze bachelor proef zoals de zonnepanelen, de laadpaal, het beluchtingssysteem van het water, … Ten slotte wordt er een vergelijking gemaakt tussen verschillende technologieën, protocollen, ... De reden voor deze vergelijking is dat er een keuze gemaakt wordt voor het creëren van een efficiënte applicatie.

\section{Zonnepanelen}
\label{sec:stand-van-zaken-zonnepanelen}

\subsection{Inleiding}
\label{sec:stand-van-zaken-zonnepanelen-inleiding}

Carwash Clean Car heeft sinds 2013 zonnepanelen liggen. Deze zonnepanelen zijn geplaatst geweest om de energiefactuur te verlagen, maar de laatste jaren merkt het bedrijf op dat de elektriciteitsfactuur terug begint te stijgen. Nu is het de bedoeling om de data van deze zonnepanelen uit te lezen, om zo de onnodige elektriciteit intensieve bedrijfsprocessen aan te sturen wanneer er genoeg elektriciteit wordt opgewekt met de zonnepanelen.
