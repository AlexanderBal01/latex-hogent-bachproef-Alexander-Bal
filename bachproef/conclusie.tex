%%=============================================================================
%% Conclusie
%%=============================================================================

\chapter{Conclusie}%
\label{ch:conclusie}

% TODO: Trek een duidelijke conclusie, in de vorm van een antwoord op de
% onderzoeksvra(a)g(en). Wat was jouw bijdrage aan het onderzoeksdomein en
% hoe biedt dit meerwaarde aan het vakgebied/doelgroep? 
% Reflecteer kritisch over het resultaat. In Engelse teksten wordt deze sectie
% ``Discussion'' genoemd. Had je deze uitkomst verwacht? Zijn er zaken die nog
% niet duidelijk zijn?
% Heeft het onderzoek geleid tot nieuwe vragen die uitnodigen tot verder 
%onderzoek?

In dit onderzoek werd er gezocht naar een antwoord op de onderzoeksvraag: “Hoe kan het bedrijf carwash Clean Car zijn maandelijkse energiefactuur verlagen door het slim aansturen van laadpalen en elektriciteit-intensieve bedrijfsprocessen aan de hand van een custom geschreven applicatie?”. Om op deze vraag een antwoord te vinden werden eerst de deelvragen beantwoord aan de hand van een literatuurstudie en een proefopstelling.\\

In de literatuurstudie zijn eerst de verschillende energieverbruikers bepaald, deze energieverbruikers kunnen dan opgedeeld worden in 2 groepen, namelijk dringende en niet-dringende bedrijfsprocessen. De dringende bedrijfsprocessen bestaan uit het aansturen van de  tunnel van de carwash en de selfcarwash. De niet-dringende bedrijfsprocessen bestaan uit de laadpaal, het beluchtingssysteem van het water en de industriële droogkast.\\

Na het bespreken van de energieverbruikers zijn er uit de literatuurstudie verschillende concepten om de energiefactuur te verlagen voortgekomen. Zo is er gesproken over Peak Shaving, bidirectioneel laden, het gebruik van algoritmen en energiebeheersystemen.\\

Het besluit over peak shaving is dat het toegepast kan worden in de proefopstelling. Hierdoor is dit ook geïmplementeerd in de applicatie. Bidirectioneel laden kan momenteel nog niet gebruikt worden, aangezien de laadpaal die momenteel geïnstalleerd staat dit nog niet ondersteunt. Het gebruik van algoritmen kan wel toegepast worden in de proefopstelling, maar kan nog verder uitgewerkt worden voor specifieke scenario’s die nog niet konden getest worden. Het energiebeheersysteem is uiteindelijk een groot onderdeel van de proefopstelling geworden, aangezien er via de applicatie alle bedrijfsprocessen aangestuurd kunnen worden, valt dit onder de categorie van een energiebeheersysteem.\\

Indien het bedrijf later toch gebruik wil maken van bidirectioneel laden, zou er ook onderzoek gedaan kunnen worden over hoe batterijen aangesloten kunnen worden voor het opslaan van energie. Met de bedoeling de opgeslagen energie later te gebruiken wanneer de energieprijs hoger is.