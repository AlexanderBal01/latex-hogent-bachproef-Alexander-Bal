%%=============================================================================
%% Samenvatting
%%=============================================================================

% TODO: De "abstract" of samenvatting is een kernachtige (~ 1 blz. voor een
% thesis) synthese van het document.
%
% Een goede abstract biedt een kernachtig antwoord op volgende vragen:
%
% 1. Waarover gaat de bachelorproef?
% 2. Waarom heb je er over geschreven?
% 3. Hoe heb je het onderzoek uitgevoerd?
% 4. Wat waren de resultaten? Wat blijkt uit je onderzoek?
% 5. Wat betekenen je resultaten? Wat is de relevantie voor het werkveld?
%
% Daarom bestaat een abstract uit volgende componenten:
%
% - inleiding + kaderen thema
% - probleemstelling
% - (centrale) onderzoeksvraag
% - onderzoeksdoelstelling
% - methodologie
% - resultaten (beperk tot de belangrijkste, relevant voor de onderzoeksvraag)
% - conclusies, aanbevelingen, beperkingen
%
% LET OP! Een samenvatting is GEEN voorwoord!

%%---------- Nederlandse samenvatting -----------------------------------------
%
% TODO: Als je je bachelorproef in het Engels schrijft, moet je eerst een
% Nederlandse samenvatting invoegen. Haal daarvoor onderstaande code uit
% commentaar.
% Wie zijn bachelorproef in het Nederlands schrijft, kan dit negeren, de inhoud
% wordt niet in het document ingevoegd.

\IfLanguageName{english}{%
\selectlanguage{dutch}
\chapter*{Samenvatting}
\lipsum[1-4]
\selectlanguage{english}
}{}

%%---------- Samenvatting -----------------------------------------------------
% De samenvatting in de hoofdtaal van het document

\chapter*{\IfLanguageName{dutch}{Samenvatting}{Abstract}}

Deze bachelorproef heeft waardevolle inzichten voor bedrijven die zich bezighouden met hun energieverbruik en dit willen optimaliseren door middel van bepaalde niet-dringende bedrijfsprocessen slim aan te sturen. Zo kunnen bedrijven hun eigen energiefactuur verlagen.\\

De overheid stimuleert bedrijven om elektrische voertuigen aan te kopen. De reden hiervoor is, dat er tegen 2026 enkel nog elektrische bedrijfsvoertuigen mogen rondrijden. Dit is gunstig voor het milieu, maar brengt ook uitdagingen met zich mee voor bedrijven en de Belgische energieleveranciers. Zo moeten de bedrijven laadpalen plaatsen om deze bedrijfsvoertuigen op te kunnen laden. Hierdoor gaan de bedrijven meer energie verbruiken aangezien de personeelsleden de auto’s tijdens de kantooruren kunnen laden. Hieruit volgt dat het energienet meer belast zal worden. Nu is de vraag van het bedrijf Carwash Clean Car of zij hun energieverbruik kunnen optimaliseren aan de hand van een custom geschreven webapplicatie? Dit onderzoek zal hier een antwoord op bieden aan de hand van een literatuurstudie, waarin verschillende aspecten aan het bod komen in verband met het optimaliseren van het elektriciteitsverbruik. Er worden verschillende algoritmes bekeken voor het prioriteren van de bedrijfsprocessen. Na de literatuurstudie wordt er een proefopstelling gemaakt om de gevonden informatie toe te passen en zo een webapplicatie te creëren voor Carwash Clean Car. Deze applicatie zal dan ook door het bedrijf benut worden, om zo het energieverbruik van bepaalde bedrijfsprocessen te optimaliseren en zo de energiekosten te verlagen.